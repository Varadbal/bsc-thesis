%----------------------------------------------------------------------------
\chapter{Theoretical Results} \label{chapter_theoreticalResults}
%----------------------------------------------------------------------------
%As it has already been stated in the previous chapters, the action language detailed in this chapter should support formal verification methods and have as many high-level control elements and data structures, as possible.
In this chapter, the various aspects of the action language are detailed. In Section \ref{section_tr_elements} the different elements and their intended semantics, in Section \ref{section_tr_syntax} the abstract and concrete syntaxes, in Section \ref{section_tr_validation} the well-formedness constraints in the form of validation rules. In Sections \ref{section_tr_lowlevel}, \ref{section_tr_xsts} and \ref{section_tr_java} the denotational semantics are presented by detailing setps of the transformation of the abstract syntax tree to various other models, namely to the xsts formalism of the Theta Framework for formal verification and Java code for execution.

As it has already been stated in previous chapters, the language is designed to support formal verification methods by remaining within the boundaries of finite-state machines, but also to have as many high-level control and data structures, as this compromise allows. It extends the type system and expression language of the Gamma Statechart Composition Framework, but it is also extendable by tool-specific elements, which results in a seamless integration.

%----------------------------------------------------------------------------
\section{Elements of the Action Language} \label{section_tr_elements}
%----------------------------------------------------------------------------
Actions - based on Definition \ref{definition_action} - are the fundamental unit of the behavior of the system. Depending on the type of the Action, it may take a set of inputs, produce a set of outputs, or modify the overall state of the system. By default, Actions are executed sequentially and can be placed in specific places in a statechart. They may have side-effects on variables and components that are located in the scope of the Action (data-flow) or influence the execution of other Actions (control-flow). Most actions are Statements, but there are also two special types of Actions: Blocks and Empty Actions.

\textbf{Assignment Statements} are Statements that enable the assignment of values to variables. The left-hand side of the assignment is a reference expression (referring to a variable declaration) and the right-hand side is an expression. After the assignment, the value of the variable referred on the left-hand side takes the evaluation of the right-hand side expression.

\textbf{Blocks} are special actions that limit the scope of other actions and also define sequential execution for the contained actions. They may contain or be contained in other actions.

\textbf{Break Statements} are Statements that can be contained in \textit{Switch Statements} and \textit{For Statements}, the behavior depending on the type of the containing Statement.

\textbf{Constant Declaration Statements} are Statements that declare and define read-only values, that can be used by other Actions. Constants possess a name, a type and a side-effect free expression for the definition of the constant. They are valid and unique in the scope in which they are declared. They are equivalent to constant declarations used  elsewhere in the Gamma Framework.

\textbf{Choice Statements} are Statements that define Blocks of actions that are executed when the guard of the block is evaluated to true. There can be an arbitrary number of guarded blocks, but at most one of them is executed. In case multiple guards are evaluated to true, one block is chosen non-deterministically.

\textbf{Expression Statements} are Statements that contain expressions. Expressions combine values and functions, and can be evaluated to a single value (called Literal Expression) of a certain type. They may have side-effects (e.g. function access expressions) - this justifies their applicability as Actions. They can be evaluated to an empty expression having the \textit{void} type, which cannot be used in other expressions. The applicable expression language is that of the Gamma Framework.

\textbf{Empty Statements} are special Statements that do not represent any behavior.

\textbf{For Statements} are Statements that define a parameter variable and a Block of actions, and take a range expression that can be evaluated to a value of a composite, enumerable type (i.e. containing multiple values). The Block is executed for each value contained in the range expression, with the parameter variable taking the corresponding value each time. The execution of the Statement can be interrupted by a \textit{Break Statement}. In case of nested For Statements, the execution of the innermost one is interrupted. There is also a possibility to define a block of actions called a \textit{then-block}, which is executed when no Break Statements have been encountered during the execution of the given For Statement.

\textbf{If Statements} are conditional Statements that define blocks of actions that are executed when the guard expression of the block is evaluated to true. There can be an arbitrary number of guarded blocks, but at most one of them is executed. In case multiple guards are evaluated to true, the earlier defined one is chosen. There is also a possibility to define a block to execute when none of the other blocks is chosen, called an \textit{else-block}.

\textbf{Return Statements} define the output values of \textit{Procedures}, also interrupting their execution.

\textbf{Switch Statements} are Statements that define blocks of actions that are executed when the evaluation of the guard expression of the block equals the evaluation of the control expression defined at the beginning of the Statement. There can be an arbitrary number of guarded blocks, and the execution starts at the first block with a guard which equals the control expression. The execution stops when a \textit{Break Statement} is reached. There is also a possibility to define a block to execute when none of the other blocks is chosen, called a \textit{default-block}, which must be the last of the blocks of the statement.

\textbf{Variable Declaration Statements} are Statements that declare and possibly define the variables on which the other Actions can operate. Variables possess a name, a type and may have a side-effect free expression for the definition of the variable. They can be read or written and are valid and unique in the scope in which they are declared. They are equivalent to variable declarations used  elsewhere in the Gamma Framework.

%----------------------------------------------------------------------------
\bigskip
The following Actions are specific to the Gamma Framework:

\textbf{Deactivate Timeout Actions} extend the action language with deactivating a timer in the Gamma Framework.

\textbf{Raise Event Actions} extend the action language by raising Events in the Gamma Framework.

\textbf{Set Timeout Actions} extend the action language with setting a timer in the Gamma Framework, that produces a timeout action after the given time interval.


%----------------------------------------------------------------------------
\section{Syntax} \label{section_tr_syntax}
%----------------------------------------------------------------------------
In this section the syntax for each of the elements is defined. It is given using the (E)BNF [TODO REF] notation. For the syntax of the action language as a whole, see [TODO].

\textbf{Assignment Statements:}
\begin{lstlisting}
	AssignmentStatement 			= AssignableExpression ':=' Expression ';' ;
	AssignableExpression 			= ReferenceExpression | ArrayAccessExpression | RecordAccessExpression ;
\end{lstlisting}
Examples:
\begin{lstlisting}
	//var myInt : integer;
	myInt := -1;
	---
	//var myArr : array integer[2];
	myArr := []{1, 2};
	myArr[1] := 3;
	---
	//var myRec : record{field1 : integer, field2 : boolean} := (# field1 := 3, field2 := false #);
	//procedure proc() : integer { return 4; }
	myRec.field1 := proc();
\end{lstlisting}

\textbf{Blocks:}
\begin{lstlisting}
	Block 			= '{' {Action} '}'
\end{lstlisting}
Examples:
\begin{lstlisting}
	{
		. . . 
	}
	---
	if(true) {
		. . .
	}
\end{lstlisting}

\textbf{Break Statements:}
\begin{lstlisting}
	BreakStatement 			= 'break' ';';
\end{lstlisting}
Examples:
\begin{lstlisting}
	for (a : integer in [0 .. 100]) {
		break;
	}
	---
	switch (true) {
		default: break;
	}
\end{lstlisting}

\textbf{Constant Declaration Statements:}
\begin{lstlisting}
	ConstantDeclarationStatement 	= 'const' ID ':' Type := Expression ';' ;
\end{lstlisting}
Examples:
\begin{lstlisting}
	const a : integer := 5;
	---
	const b : array integer[2] := []{1, 2}; 
\end{lstlisting}

\textbf{Choice Statements:}
\begin{lstlisting}
	ChoiceStatement 		= 'choice' '{'
		'branch' '[' Expression ']' Action
		{'branch' '[' Expression ']' Action}
	'}'	;
\end{lstlisting}
Example:
\begin{lstlisting}
	choice {
		branch [a < 1] {
			. . .
		}
		branch [a < 2] {
			. . .
		}
		branch [true] {
			. . .
		}
	}
\end{lstlisting}

\textbf{Expression Statements:}
\begin{lstlisting}
	ExpressionStatement 			= Expression ';' ;		//based on the expression language of Gamma
\end{lstlisting}
Examples:
\begin{lstlisting}
	1;						//integer literal expression
	---
	1 + 1;					//addition expression
	---
	5 div 2;				//integer division expression
	---
	1 < 2;					//less expression
	---
	true = true;			//equality expression
	---
	myArray[2];				//array access expression
	---
	myFunction(1, true);	//function access expression
\end{lstlisting}

\textbf{Empty Statements:}
Empty Statements do not have a textual representation.

\textbf{For Statements:}
\begin{lstlisting}
	ForStatement 		= 	'for' '(' ParameterDeclaration ':' Expression ')'
			Action
		'then'
			Action;
	ParameterDeclaration = ID ':' Type ;	
\end{lstlisting}
Examples:
\begin{lstlisting}
	//var myArr : array integer[3] := []{1, 2, 3};
	for(p : myArr){
		. . .
	}
	---
	//var myArr : array integer[5] := []{1, 2, 3, 4, 5};
	//var myArr2 : array integer[4];
	//var max := 3;
	for(p : [0 .. 4]) {
		myArr2[p] := myArr[p];
		if(p > max)
			break;
	} then {
		raise Events.NoError;	//assuming the Events.NoError event exists
	}
\end{lstlisting}

\textbf{If Statements:}
\begin{lstlisting}
	IfStatement 		= 'if' '(' Expression ')' Action
										{'elsif' '(' Expression ')' Action}
										['else' Action]
										;
\end{lstlisting}
Examples:
\begin{lstlisting}
	if (a /= b) {
		. . .
	}
	---
	if (a < b) {
		. . .
	} elsif (a > b) {
		. . .
	} else {
		. . .
	}
	---
	//in a For Statement:
	{
		if(true)
			break;
	}
\end{lstlisting}

\textbf{Return Statements:}
\begin{lstlisting}
	ReturnStatement 			= 'return' Expression ';';
\end{lstlisting}
Examples:
\begin{lstlisting}
	procedure proc() : void {
		return;
	}
	---
	procedure proc2() : integer{
		for(a : integer in [0 .. 100]){
			if(a = 55)
				return a;
		}
	}
\end{lstlisting}

\textbf{Switch Statements} are....
\begin{lstlisting}
	SwitchStatement 		= 'switch' '(' Expression ')' '{'
			'case' Expression ':' Action
			{'case'  Expression ':' Action}
			['default' ':' Action]
		'}'	;
\end{lstlisting}
Examples:
\begin{lstlisting}
	switch (a) {
		case b : {		
			. . .
		}
		case c : {		
			. . .
			break;
		}
		case d : break;	
		deafult: {		
			. . .
		}
	}
\end{lstlisting}

\textbf{Variable Declaration Statements:}
\begin{lstlisting}
	//based on the variables of Gamma:
	VariableDeclarationStatement 	= 'var' ID ':' Type [':=' Expression] ';' ;	
\end{lstlisting}
Examples:
\begin{lstlisting}
	var a : integer;
	---
	var b : boolean := true;
	---
	var c : array integer[3];
\end{lstlisting}

%----------------------------------------------------------------------------
\bigskip
The following Actions are specific to the Gamma Framework:

\textbf{Deactivate Timeout Actions:}
\begin{lstlisting}
	DeactivateTimeoutAction 	= 'deactivate' TimeoutDeclaration ;
\end{lstlisting}
Examples:
\begin{lstlisting}
	//timeout timeOut
	deactivate timeOut;
\end{lstlisting}

\textbf{Raise Event Actions:}
\begin{lstlisting}
	RaiseEventAction 		= 'raise' Port '.' Event '(' [Expression {',' Expression}] ')'  ;	
\end{lstlisting}
Examples:
\begin{lstlisting}
	//Toggle is a provided interface with the out event toggle
	raise Toggle.toggle;
	---
	//Operator is a provided interface with the out event add(p : integer)
	raise Operator.add(5);
\end{lstlisting}

\textbf{Set Timeout Actions:}
\begin{lstlisting}
	SetTimeoutAction 			= 'set' TimeoutDeclaration ':=' TimeSpecification ;	
	TimeSpecification 			= AdditiveExpression TimeUnit
	TimeUnit 					= 's' | 'ms'
\end{lstlisting}
Examples:
\begin{lstlisting}
	//timeout timeOut
	set timeOut := 5 s;
	---
	//timeout timeOut
	//var offset : integer := 500
	set timeOut := 100 + offset ms;
\end{lstlisting}

%----------------------------------------------------------------------------
\bigskip
Functions are not Actions themselves, but procedure declarations contain Actions:

\textbf{Functions:}
\begin{lstlisting}
	FunctionDeclaration 		= LambdaDeclaration | ProcedureDeclaration ;
	LambdaDeclaration 			= 'lambda' ID '(' [ParameterDeclaration {',' ParameterDeclaration}] ')' 
		':' Type ':=' [Expression] ;
	ProcedureDeclaration		= 'procedure' ID '(' [ParameterDeclaration {',' ParameterDeclaration}] ')' 
		':' Type Block ;

\end{lstlisting}
Examples:
\begin{lstlisting}
	procedure myProc() : void {
		. . .
	}
	---
	procedure myProc2(a : integer, b : boolean) : integer {
		. . .
		return a + 5;
	}
\end{lstlisting}

//TODO FULL BNF

%----------------------------------------------------------------------------
\section{Validation} \label{section_tr_validation}
%----------------------------------------------------------------------------
Validation rules enforce the well-formedness constraints of the modeling language in the actual development environment. Some of these only warn the user of potential design flaws -- so they can be discovered as early in the design phase, as possible -- while others mark non-deterministic or faulty behavior or an untransformable combination of Actions in the form of errors.

\subsection{Warnings}

\subsection{Errors}

%----------------------------------------------------------------------------
\section{Low-level model} \label{section_tr_lowlevel}
%----------------------------------------------------------------------------
In order to transform our high-level models -- that are easy to understand, handle and explain, with several language elements and their complex relations -- into various other (often formal) models, it is useful to transform them into one with only a few different types of elements and many basic relations. For statecharts, this means substituting syntactic elements with one resembling a finite-state machine: several transitions instead of hierarchical states or the product of the states of concurrent regions. These few simple constructs are then easily transformable into elements of other modeling languages

For action languages, this means reducing the number of available actions, as well as expressions and types, by transforming the composite elements into constructs using only basic ones. Naturally, it is neither sensible nor possible to reduce the number of applicable elements indefinitely, but finding the elements resembling those of most other formalisms greatly simplifies the further transformations. 

In the rest of the section, these high-level - low-level transformations of the language elements are examined:

\textbf{Variable Declaration Statements} are....
//Expression language VariableDeclaration

\textbf{Constant Declaration Statements} are....
//Expression language ConstantDeclaration

\textbf{Expression Statements} are....
//EXPRESSION LANGUAGE HERE???

\textbf{Assignment Statements} are....

\textbf{If Statements} are....
Elsif, else

\textbf{Choice Statements} are....

\textbf{Switch Statements} are....
//break...

\textbf{For Statements} are....
//Then
//break

\textbf{Functions} are....
\textbf{Procedures}
//return

%----------------------------------------------------------------------------
//Statechart Extensions

\textbf{Set Timeout Actions} are....

\textbf{Deactivate Timeout Actions} are....

\textbf{Raise Event Actions} are....

%----------------------------------------------------------------------------
\section{XSTS} \label{section_tr_xsts}
%----------------------------------------------------------------------------
For the formal verification of the designed systems, currently the transformation to the XSTS formalism of the Theta Framework is supported. This formalism contains very low-level constructs: only variables, assignments, assumptions and conditional and parallel branching of the control flow are supported, with an expression language only supporting basic mathematical and logic expressions. 

\subsection{XSTS syntax}
//Choice

//Assume

//Parallel

//Expressions

\subsection{XSTS syntax}


\subsection{Transformation of the low-level action language elements}


%----------------------------------------------------------------------------
\section{Executable code} \label{section_tr_java}
%----------------------------------------------------------------------------


%----------------------------------------------------------------------------
\section{Integration into the Gamma Framework} \label{section_tr_integration}
%----------------------------------------------------------------------------
//This section summarizes the previous ones

//(relevant) Components of the Gamma Framework (expression-action-statechart -> redExpression-redAction-lowlevelSC -> xsts | Java)