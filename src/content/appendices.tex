%----------------------------------------------------------------------------
\appendix
%----------------------------------------------------------------------------
%\chapter*{\fuggelek}\addcontentsline{toc}{chapter}{\fuggelek}
%\setcounter{chapter}{\appendixnumber}
%\setcounter{equation}{0} % a fofejezet-szamlalo az angol ABC 6. betuje (F) lesz
%\numberwithin{equation}{section}
%\numberwithin{figure}{section}
%\numberwithin{lstlisting}{section}
%\numberwithin{tabular}{section}

\chapter{Action Language}
%----------------------------------------------------------------------------
\section{Overview of the Action Language}
%----------------------------------------------------------------------------
\bigskip
\begin{lstlisting} [language=tex,caption=Overivew of the syntax of the action language,label=lst:SCActionLanguageOverview]
	Action = Statement | Block ;
	
	Block = '{'
						{Action}
					'}' 
					;
					
	Statement = VariableDeclarationStatement |
							ConstantDeclarationStatement | 
							ExpressionStatement |
							AssignmentStatement |
							RaiseEventAction |
							SetTimeoutDeclaration |
							DeactivateTimeoutDeclaration |
							IfStatement |
							ChoiceStatement |
							ReturnStatement |
							BreakStatement |
							SwitchStatement |
							ForStatement |
							AssertionStatement
							;		
\end{lstlisting}

%----------------------------------------------------------------------------
\chapter{Model Transformations}
%----------------------------------------------------------------------------
\newpage
%----------------------------------------------------------------------------
\section{Overview of the High-level-to-Low-level Transformation}
%----------------------------------------------------------------------------
\begin{table}[H]
	%\footnotesize
	\centering
	\renewcommand{\arraystretch}{1.5}
	\begin{tabular}{ | l | l | }
		\toprule
		\textbf{High-level} & \textbf{Low-level} \\
		\hline
		\midrule
		Variable Declaration Statement & Variable Declaration Statement(s) \\ \hline
		Constant Declaration Statement & Variable Declaration Statement(s) \\ \hline
		Parameter Declaration & Variable Declaration and Assignment Statement(s) \\ \hline
		Expression Statement & Empty Action or Variable Declaration, Assignment, \\
		& possibly If and Choice Statements\\ \hline
		Reference Expression & (simple) Reference Expression or Variable \\ 
		& Declaration, Assignment and If Statement(s)\\ \hline
		Block & Block \\ \hline
		If Statement & Variable Declarations, Assignments and \\ 
		& an If Statement  \\ \hline
		Choice Statement & Variable Declarations, Assignments and \\
		& a Choice Statement \\ \hline
		Break Statement & -- \\ \hline
		Return Statement & Assignment Statement or -- \\ \hline
		Switch Statement & Variable Declarations, Assignments and \\
		& an If Statement \\ \hline
		For Statement & Variable Declarations, Assignments  \\ 
		& (possibly If or Choice Statements)\\ \hline
		Set Timeout Action & Assignment Statement \\ \hline
		Deactivate Timeout Action & Not yet transformed \\ \hline
		Raise Event Action & Assignment Statement(s) \\ \hline
		Empty Statement & Empty Statement \\ \hline
		\bottomrule
	\end{tabular}
	\caption{Overview of the high-level-to-low-level action transformation}
	\label{tab:SCLLOverview}
\end{table}

%----------------------------------------------------------------------------
\section{Overview of the Low-level-to-xSTS Transformation}
%----------------------------------------------------------------------------
\begin{table}[H]
	%\footnotesize
	\centering
	\renewcommand{\arraystretch}{1.5}
	\begin{tabular}{ | l | l |}
		\toprule
		\textbf{High-level Element} & \textbf{xSTS Element(s)} \\ \hline
		\midrule
		Variable Declaration Statement & Empty Statement, temporary Variable Declaration \\ \hline
		Reference Expression & Reference Expression \\ \hline
		Variable Initializing Expression & Variable Initializing Expression \\ \hline
		Assignment Statement & Assignment Action\\ \hline
		Block & Sequential Action \\ \hline
		If Statement & Non-deterministic Action with Sequential Action(s) \\
		& and Assume Action(s) \\ \hline
		Choice Statement & Non-deterministic Action and Sequential Action(s) \\
		& and Assume Actions(s) \\ \hline
		Empty Statement & Empty Statement \\ \hline
		\bottomrule
	\end{tabular}
	\caption{Overview of the low-level statechart-to-xSTS transformation}
	\label{tab:LLXSTSOverview}
\end{table}

%----------------------------------------------------------------------------
\chapter{Validation and Case Study}
%----------------------------------------------------------------------------
\newpage
%----------------------------------------------------------------------------
\section{Testing the Language Elements}
%----------------------------------------------------------------------------
\begin{table}[H]
	\footnotesize
	\centering
	\begin{tabular}{ l l l }
		\toprule
		Test Suite & Test Case & Result \\
		\midrule
		Raise Events & Test one simple event & \textcolor{green}{Passed}\\
		& Test two simple events & \textcolor{green}{Passed}\\
		& Test one parametric event & \textcolor{green}{Passed}\\
		Variable Declarations & test one integer variable (with initialization) & \textcolor{green}{Passed}\\
		& Test one boolean variable (with initialization) & \textcolor{green}{Passed}\\
		& Test one integer array variable (with initialization) & \textcolor{green}{Passed}\\
		& Test one boolean variable (with initialization) & \textcolor{green}{Passed}\\
		& Test one record variable (with initialization) & \textcolor{green}{Passed}\\
		& Test one enum variable (with initialization) & \textcolor{green}{Passed}\\
		Constant Declarations & Same as the cases of variable declarations & \textcolor{gray}{Not tested}\\
		Expression Statements & Either at the procedures or cannot be tested & \textcolor{gray}{Not tested}\\
		Assignment Statements & Test \textit{integer reference := integer literal} & \textcolor{green}{Passed}\\
		& Test \textit{integer reference := integer reference} & \textcolor{green}{Passed}\\
		& Test \textit{integer reference := integer function} & \textcolor{red}{Failed}\\
		& Test \textit{integer reference := record-integer} & \textcolor{green}{Passed}\\
		& Test \textit{integer reference := array-integer} & \textcolor{green}{Passed}\\
		& Test \textit{array-integer := integer reference} & \textcolor{green}{Passed}\\
		& Test \textit{record-integer := integer reference} & \textcolor{green}{Passed}\\
		If Statements & Test \textit{if(true)} & \textcolor{green}{Passed}\\
		& Test \textit{if(false)} & \textcolor{green}{Passed}\\
		& Test \textit{if(true reference)} & \textcolor{green}{Passed}\\
		& Test \textit{if(false reference)} & \textcolor{green}{Passed}\\
		& Test \textit{if(true)else} & \textcolor{green}{Passed}\\
		& Test \textit{if(false)else} & \textcolor{green}{Passed}\\
		& Test \textit{if(true)elsif(true)} & \textcolor{green}{Passed}\\
		& Test \textit{if(false)elsif(false)} & \textcolor{green}{Passed}\\
		& Test \textit{if(false)elsif(true)} & \textcolor{gray}{Not tested}\\
		& Test \textit{if(true){if(true)}} & \textcolor{green}{Passed}\\
		& Test \textit{if(true)if(true)} & \textcolor{green}{Passed}\\
		Choice Statements & Test branch(true) & \textcolor{green}{Passed}\\
		& Test branch(false) & \textcolor{red}{Failed}\\
		& Test branch(true)branch(true) & \textcolor{green}{Passed}\\
		& Test branch(false)branch(true) & \textcolor{green}{Passed}\\
		Switch Statements & Test case(true) & \textcolor{green}{Passed}\\
		& Test case(false) & \textcolor{green}{Passed}\\
		& Test case(true)case(false) & \textcolor{green}{Passed}\\
		& Test case(false)case(true) & \textcolor{green}{Passed}\\
		& Test case(true)breakCase(false) & \textcolor{green}{Passed}\\
		For Statements & Test for array reference, only for\{\} & \textcolor{green}{Passed}\\
		& Test for array reference, for\{break\}then\{\} & \textcolor{green}{Passed}\\
		& Test for array reference, for\{noBreak\}then\{\} & \textcolor{green}{Passed}\\
		& Test for array reference, accessing the parameter values & \textcolor{green}{Passed}\\
		\bottomrule
	\end{tabular}
	\caption{Results of the validation of the individual language elements}
	\label{tab:ValidationTable}
\end{table}

\newpage
%----------------------------------------------------------------------------
\section{RPN Calculator}
%----------------------------------------------------------------------------

\begin{lstlisting} [language=tex,caption=The original RPN left-to-right algirithm,label=lst:RPNLeftToRight]
	for each token in the postfix expression:
		if token is an operator:
			operand_2 ← pop from the stack
			operand_1 ← pop from the stack
			result ← evaluate token with operand_1 and operand_2
			push result back onto the stack
		else if token is an operand:
			push token onto the stack
	result ← pop from the stack
\end{lstlisting}

\bigskip
\begin{lstlisting} [language=tex,caption=Interfaces of the RPN calculator,label=lst:RPNCalculatorInterfacesSource]
	package Interfaces
	
	interface CalcInput {
		out event operator (inOperator : Operator)
		out event operand (inOperand : integer)
		out event evaluate
	}
	
	interface CalcOutput {
		out event success (result : integer)
		out event error
	}
	
	type Operator : enum {
		addition, subtraction, multiplication, division
	}
\end{lstlisting}

\bigskip
\begin{lstlisting} [language=tex,caption=Model of the RPN calcuator,label=lst:RPNCalculatorModelSource]
	package Calculator
	
	import "Interfaces"
	
	statechart CalculatorStatechart[
			port Input : requires CalcInput
			port Output : provides CalcOutput
		]{
		// Variables
		var operators : array Operator[2]
		var numOperators : integer := 0
		var operands : array integer[2]
		var numOperands : integer := 0
		var isOperator : array boolean[4]
		var numIsOperator : integer := 0	//expression length
		
		
		// Transitions
		transition from calcEntry to Init
		
		transition from Init to OneNumber when Input.operand / {
			operands[numOperands] := Input.operand::inOperand;
			isOperator[numIsOperator] := false;
			numOperands := numOperands + 1;
			numIsOperator := numIsOperator + 1;
		}
		transition from Init to Init when Input.operator / {
			raise Output.error;
		}
		transition from Init to Init when Input.evaluate / {
			raise Output.error;
		}
		
		transition from OneNumber to Invalid when Input.operand / {
			operands[numOperands] := Input.operand::inOperand;
			isOperator[numIsOperator] := false;
			numOperands := numOperands + 1;
			numIsOperator := numIsOperator + 1;
		}
		transition from OneNumber to Init when Input.operator / {
			raise Output.error;
			numOperators := 0;
			numOperands := 0;
			numIsOperator := 0;
		}
		transition from OneNumber to Init when Input.evaluate / {
			var oneNumberResult : integer;
			oneNumberResult := operands[0];
			raise Output.success(oneNumberResult);
			numOperands := 0;
			numIsOperator := 0;
		}
		
		transition from Invalid to Invalid when Input.operand / {
			operands[numOperands] := Input.operand::inOperand;
			isOperator[numIsOperator] := false;
			numOperands := numOperands + 1;
			numIsOperator := numIsOperator + 1;
		}
		transition from Invalid to MaybeValid when Input.operator / {
			operators[numOperators] := Input.operator::inOperator; 
			isOperator[numIsOperator] := true;
			numOperators := numOperators + 1;
			numIsOperator := numIsOperator + 1;
		}
		transition from Invalid to Init when Input.evaluate / {
			raise Output.error; 
			numOperators := 0; 
			numOperands := 0;
			numIsOperator := 0;
		}
		
		transition from MaybeValid to Invalid when Input.operand / {
			operands[numOperands] := Input.operand::inOperand;
			isOperator[numIsOperator] := false;
			numOperands := numOperands + 1;
			numIsOperator := numIsOperator + 1;
		}
		transition from MaybeValid to MaybeValid when Input.operator / {
			operators[numOperators] := Input.operator::inOperator; 
			isOperator[numIsOperator] := true;
			numOperators := numOperators + 1;
			numIsOperator := numIsOperator + 1;
		}
		transition from MaybeValid to Init when Input.evaluate [numOperators = numOperands - 1] / {
			var finalResult : integer; 
			
			// left-to-right algorithm
			var resultStack : array integer[2];
			
			var stackTop : integer := 0;
			var operatorsTop : integer := 0;
			var operandsTop : integer := 0;
			var isOperatorValue : boolean;
			
			//variables in the for loop
			var temp0 : Operator;
			var temp1 : integer;
			var temp2 : integer;
			var temp : integer;
			
			for (i : integer in [0 .. 4]){
				if (i < (numOperators + numOperands)) {
					isOperatorValue := isOperator[i];
					if (isOperatorValue) {
						// Get values
						temp0 := operators[operatorsTop]; operatorsTop := operatorsTop + 1;
						temp1 := resultStack[stackTop]; stackTop := stackTop - 1;
						temp2 := resultStack[stackTop]; stackTop := stackTop - 1;
						
						// Calculate
						if (temp0 = ::addition) {
							resultStack[stackTop] := temp1 + temp2; stackTop := stackTop - 1;
						}
						else if (temp0 = ::subtraction) {
							resultStack[stackTop] := temp1 - temp2; stackTop := stackTop - 1;
						}
						else if (temp0 = ::multiplication) {
							resultStack[stackTop] := temp1 * temp2; stackTop := stackTop - 1;
						}
						else if (temp0 = ::division) {
							resultStack[stackTop] := temp1 div temp2; stackTop := stackTop - 1;
						}
						
					}
					else {
						temp := operands[operandsTop]; operandsTop := operandsTop + 1;
						resultStack[stackTop] := temp; stackTop := stackTop + 1;
					}
				}
				else {
					break;
				}
			}
			
			//
			finalResult := resultStack[0];
			raise Output.success(finalResult);
			
			//
			numOperators := 0;
			numOperands := 0;
			numIsOperator := 0;
		}
		transition from MaybeValid to Init when Input.evaluate [numOperators /= numOperands - 1] / {
			raise Output.error; 
			numOperators := 0; 
			numOperands := 0;
			numIsOperator := 0;
		}
		
		// States
		region mainRegion{
			initial calcEntry
			state Init{
			
			}
			state OneNumber{
			
			}
			state Invalid{
			
			}
			state MaybeValid{
			
			}
		}
	}
\end{lstlisting}
